% Options for packages loaded elsewhere
\PassOptionsToPackage{unicode}{hyperref}
\PassOptionsToPackage{hyphens}{url}
\PassOptionsToPackage{dvipsnames,svgnames,x11names}{xcolor}
%
\documentclass[
  letterpaper,
  DIV=11,
  numbers=noendperiod]{scrartcl}

\usepackage{amsmath,amssymb}
\usepackage{iftex}
\ifPDFTeX
  \usepackage[T1]{fontenc}
  \usepackage[utf8]{inputenc}
  \usepackage{textcomp} % provide euro and other symbols
\else % if luatex or xetex
  \usepackage{unicode-math}
  \defaultfontfeatures{Scale=MatchLowercase}
  \defaultfontfeatures[\rmfamily]{Ligatures=TeX,Scale=1}
\fi
\usepackage{lmodern}
\ifPDFTeX\else  
    % xetex/luatex font selection
\fi
% Use upquote if available, for straight quotes in verbatim environments
\IfFileExists{upquote.sty}{\usepackage{upquote}}{}
\IfFileExists{microtype.sty}{% use microtype if available
  \usepackage[]{microtype}
  \UseMicrotypeSet[protrusion]{basicmath} % disable protrusion for tt fonts
}{}
\makeatletter
\@ifundefined{KOMAClassName}{% if non-KOMA class
  \IfFileExists{parskip.sty}{%
    \usepackage{parskip}
  }{% else
    \setlength{\parindent}{0pt}
    \setlength{\parskip}{6pt plus 2pt minus 1pt}}
}{% if KOMA class
  \KOMAoptions{parskip=half}}
\makeatother
\usepackage{xcolor}
\setlength{\emergencystretch}{3em} % prevent overfull lines
\setcounter{secnumdepth}{-\maxdimen} % remove section numbering
% Make \paragraph and \subparagraph free-standing
\ifx\paragraph\undefined\else
  \let\oldparagraph\paragraph
  \renewcommand{\paragraph}[1]{\oldparagraph{#1}\mbox{}}
\fi
\ifx\subparagraph\undefined\else
  \let\oldsubparagraph\subparagraph
  \renewcommand{\subparagraph}[1]{\oldsubparagraph{#1}\mbox{}}
\fi


\providecommand{\tightlist}{%
  \setlength{\itemsep}{0pt}\setlength{\parskip}{0pt}}\usepackage{longtable,booktabs,array}
\usepackage{calc} % for calculating minipage widths
% Correct order of tables after \paragraph or \subparagraph
\usepackage{etoolbox}
\makeatletter
\patchcmd\longtable{\par}{\if@noskipsec\mbox{}\fi\par}{}{}
\makeatother
% Allow footnotes in longtable head/foot
\IfFileExists{footnotehyper.sty}{\usepackage{footnotehyper}}{\usepackage{footnote}}
\makesavenoteenv{longtable}
\usepackage{graphicx}
\makeatletter
\def\maxwidth{\ifdim\Gin@nat@width>\linewidth\linewidth\else\Gin@nat@width\fi}
\def\maxheight{\ifdim\Gin@nat@height>\textheight\textheight\else\Gin@nat@height\fi}
\makeatother
% Scale images if necessary, so that they will not overflow the page
% margins by default, and it is still possible to overwrite the defaults
% using explicit options in \includegraphics[width, height, ...]{}
\setkeys{Gin}{width=\maxwidth,height=\maxheight,keepaspectratio}
% Set default figure placement to htbp
\makeatletter
\def\fps@figure{htbp}
\makeatother

\KOMAoption{captions}{tableheading}
\makeatletter
\makeatother
\makeatletter
\makeatother
\makeatletter
\@ifpackageloaded{caption}{}{\usepackage{caption}}
\AtBeginDocument{%
\ifdefined\contentsname
  \renewcommand*\contentsname{Table of contents}
\else
  \newcommand\contentsname{Table of contents}
\fi
\ifdefined\listfigurename
  \renewcommand*\listfigurename{List of Figures}
\else
  \newcommand\listfigurename{List of Figures}
\fi
\ifdefined\listtablename
  \renewcommand*\listtablename{List of Tables}
\else
  \newcommand\listtablename{List of Tables}
\fi
\ifdefined\figurename
  \renewcommand*\figurename{Figure}
\else
  \newcommand\figurename{Figure}
\fi
\ifdefined\tablename
  \renewcommand*\tablename{Table}
\else
  \newcommand\tablename{Table}
\fi
}
\@ifpackageloaded{float}{}{\usepackage{float}}
\floatstyle{ruled}
\@ifundefined{c@chapter}{\newfloat{codelisting}{h}{lop}}{\newfloat{codelisting}{h}{lop}[chapter]}
\floatname{codelisting}{Listing}
\newcommand*\listoflistings{\listof{codelisting}{List of Listings}}
\makeatother
\makeatletter
\@ifpackageloaded{caption}{}{\usepackage{caption}}
\@ifpackageloaded{subcaption}{}{\usepackage{subcaption}}
\makeatother
\makeatletter
\@ifpackageloaded{tcolorbox}{}{\usepackage[skins,breakable]{tcolorbox}}
\makeatother
\makeatletter
\@ifundefined{shadecolor}{\definecolor{shadecolor}{rgb}{.97, .97, .97}}
\makeatother
\makeatletter
\makeatother
\makeatletter
\makeatother
\ifLuaTeX
  \usepackage{selnolig}  % disable illegal ligatures
\fi
\IfFileExists{bookmark.sty}{\usepackage{bookmark}}{\usepackage{hyperref}}
\IfFileExists{xurl.sty}{\usepackage{xurl}}{} % add URL line breaks if available
\urlstyle{same} % disable monospaced font for URLs
\hypersetup{
  pdftitle={Exploring reprex: A Reproducible Example Engine for R},
  colorlinks=true,
  linkcolor={blue},
  filecolor={Maroon},
  citecolor={Blue},
  urlcolor={Blue},
  pdfcreator={LaTeX via pandoc}}

\title{Exploring reprex: A Reproducible Example Engine for R}
\author{Mubarak Mojoyinola \and Ethan Murra \and Max Miller \and Yash
Vora}
\date{}

\begin{document}
\maketitle
\ifdefined\Shaded\renewenvironment{Shaded}{\begin{tcolorbox}[frame hidden, enhanced, boxrule=0pt, interior hidden, breakable, sharp corners, borderline west={3pt}{0pt}{shadecolor}]}{\end{tcolorbox}}\fi

\begin{center}\rule{0.5\linewidth}{0.5pt}\end{center}

\hypertarget{introduction-to-reproducibility}{%
\subsection{Introduction to
Reproducibility}\label{introduction-to-reproducibility}}

\begin{itemize}
\tightlist
\item
  Define reproducibility in the context of data analysis and programming
\item
  Highlight the importance of reproducibility in research and data
  science
\end{itemize}

\begin{center}\rule{0.5\linewidth}{0.5pt}\end{center}

\hypertarget{motivation-for-reprex}{%
\subsection{Motivation for reprex}\label{motivation-for-reprex}}

\begin{itemize}
\tightlist
\item
  Discuss common challenges faced in creating reproducible examples in R
\item
  Introduce reprex as a solution to these challenges
\end{itemize}

\begin{center}\rule{0.5\linewidth}{0.5pt}\end{center}

\hypertarget{what-is-reprex}{%
\subsection{What is reprex?}\label{what-is-reprex}}

\begin{itemize}
\tightlist
\item
  Definition and purpose of reprex
\item
  Briefly mention the package's history and development
\end{itemize}

\begin{center}\rule{0.5\linewidth}{0.5pt}\end{center}

\hypertarget{installation-and-setup}{%
\subsection{Installation and Setup}\label{installation-and-setup}}

\begin{itemize}
\tightlist
\item
  Instructions for installing reprex
\item
  Brief demonstration of package setup and configuration
\end{itemize}

\begin{center}\rule{0.5\linewidth}{0.5pt}\end{center}

\hypertarget{reprex-workflow}{%
\subsection{reprex Workflow}\label{reprex-workflow}}

\begin{itemize}
\tightlist
\item
  Overview of the typical workflow when using reprex
\item
  Steps involved in creating a reproducible example
\end{itemize}

\begin{center}\rule{0.5\linewidth}{0.5pt}\end{center}

\hypertarget{creating-a-basic-reprex}{%
\subsection{Creating a Basic Reprex}\label{creating-a-basic-reprex}}

\begin{itemize}
\tightlist
\item
  Demonstrate how to create a simple reprex using a basic R script
\item
  Emphasize the key components, including code, input, and output
\end{itemize}

\begin{center}\rule{0.5\linewidth}{0.5pt}\end{center}

\hypertarget{advanced-reprex-features}{%
\subsection{Advanced reprex Features}\label{advanced-reprex-features}}

\begin{itemize}
\tightlist
\item
  Explore advanced options, such as rendering to R Markdown or saving as
  a standalone file
\item
  Discuss how these features enhance the reproducibility process
\end{itemize}

\begin{center}\rule{0.5\linewidth}{0.5pt}\end{center}

\hypertarget{handling-external-dependencies}{%
\subsection{Handling External
Dependencies}\label{handling-external-dependencies}}

\begin{itemize}
\tightlist
\item
  Explain how reprex handles dependencies, such as packages and data
  files
\item
  Provide examples of including necessary resources in a reprex
\end{itemize}

\begin{center}\rule{0.5\linewidth}{0.5pt}\end{center}

\hypertarget{customizing-reprex-behavior}{%
\subsection{Customizing reprex
Behavior}\label{customizing-reprex-behavior}}

\begin{itemize}
\tightlist
\item
  Discuss options for customizing the behavior of reprex
\item
  Illustrate how to set specific parameters to meet individual
  requirements
\end{itemize}

\begin{center}\rule{0.5\linewidth}{0.5pt}\end{center}

\hypertarget{sharing-and-publishing-reprex}{%
\subsection{Sharing and Publishing
Reprex}\label{sharing-and-publishing-reprex}}

\begin{itemize}
\tightlist
\item
  Walk through the process of sharing a reprex with others
\item
  Discuss best practices for publishing reprex examples online
\end{itemize}

\begin{center}\rule{0.5\linewidth}{0.5pt}\end{center}

\hypertarget{reprex-in-data-science-workflows}{%
\subsection{reprex in Data Science
Workflows}\label{reprex-in-data-science-workflows}}

\begin{itemize}
\tightlist
\item
  Highlight how reprex fits into the broader data science workflow
\item
  Showcase real-world examples of how it has improved reproducibility in
  projects
\end{itemize}

\begin{center}\rule{0.5\linewidth}{0.5pt}\end{center}

\hypertarget{use-cases-and-examples}{%
\subsection{Use Cases and Examples}\label{use-cases-and-examples}}

\begin{itemize}
\tightlist
\item
  Provide concrete examples of scenarios where reprex is particularly
  useful
\item
  Include before-and-after comparisons to highlight its impact on
  reproducibility
\end{itemize}

\begin{center}\rule{0.5\linewidth}{0.5pt}\end{center}

\hypertarget{best-practices-and-tips}{%
\subsection{Best Practices and Tips}\label{best-practices-and-tips}}

\begin{itemize}
\tightlist
\item
  Share recommended practices for creating effective and informative
  reprex examples
\item
  Offer tips for optimizing the use of reprex in different contexts
\end{itemize}

\begin{center}\rule{0.5\linewidth}{0.5pt}\end{center}

\hypertarget{conclusion-and-future-directions}{%
\subsection{Conclusion and Future
Directions}\label{conclusion-and-future-directions}}

\begin{itemize}
\tightlist
\item
  Summarize the key takeaways from the presentation
\item
  Discuss potential future developments and improvements for reprex
\end{itemize}

\begin{center}\rule{0.5\linewidth}{0.5pt}\end{center}



\end{document}
